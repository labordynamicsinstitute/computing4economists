% Time-stamp: <09/10/02 01:57:13 vilhuber>
% $Id: Presentation-PSD.tex 3219 2012-09-27 07:47:11Z vilhu001 $

% normal line:
\documentclass[xcolor=table,compress]{beamer}
% to create notes:
%\documentclass[handout,notes=only]{beamer}
% to create handouts
%\documentclass[xcolor=table,handout,compress]{beamer}
% to create a different kind of handouts
%\documentclass{article}
%\usepackage[envcountsect]{beamerarticle}

%\setbeameroption{handout}
%\setbeameroption{show notes}


%
% Packages
%
\mode<article> % only for the article version
{
  \usepackage{fullpage}
  \usepackage{hyperref}
}
\usepackage{ifpdf}
\ifpdf
\usepackage{embedfile}
\embedfile{\jobname.tex}
\fi

\usepackage{graphicx}
%\usepackage{pstricks}
\usepackage{xcolor}
\usepackage{pifont}
%\usepackage{../chicago}
\usepackage{pgf}
\usepackage{amsmath,amssymb,amsfonts}
\usepackage[latin1]{inputenc}
\usepackage{colortbl}
\usepackage[english]{babel}
\usepackage{array}
\usepackage{pdfpages}
% usage:
%   \includepdf[pages={1}]{myfile.pdf}
%   \includepdf[pages={1,3,5}]{myfile.pdf} would include pages 1, 3, and 5 of the file. 
%   To include the entire file, you specify pages={-}, where {-}
%\usepackage{landscape}

%\usepackage{lmodern}
%\usepackage[T1]{fontenc}

\usepackage{times}
%\usepackage{colortbl}

%============================================================
% Beamer specific styles and configs
%============================================================

\mode<presentation>
{
% alternative, could always use
%\usetheme{Census}
\usetheme{cornell}
\useoutertheme{cornell}
}


%\setbeamercovered{dynamic}



%============================================================
% Title
%============================================================

\title[Computing for Economists]{Workshop: High-performance computing for economists}
\author[Vilhuber, Abowd, Mansfield, McKinney]{%
  Lars~Vilhuber\inst{1} \and
  John M. Abowd\inst{1} \and
  Richard~Mansfield\inst{1} \and
  Kevin~L.~McKinney %\inst{2}%
}

\institute[Cornell]{
  \inst{1}%
   Cornell University, Economics Department,
%\and \inst{2} U.S. Census Bureau
}%
\date[August 20-22, 2013]{August 20-22, 2013: Day 1}
\subject{HPC}


% % % % % % % % % % % % % % % % % Main document
\begin{document}
\frame{\titlepage}
\section{Intro}
\section{Basics}
\section[VCS]{Version control systems}
\section{Subroutines}

\subsection{Goals}
\begin{frame}{Basic subroutine programming}
\begin{block}{Goal}
\begin{itemize}
\item Show the basics of proper subroutine programming
\item Advantages, pitfalls
\item Examples in R
\item Tomorrow: generalization and differences in other programming languages
\end{itemize}
\end{block}
\end{frame}
\subsection[Control]{Control structures}
\begin{frame}{Control structures in programming languages}
\small
\begin{block}{Mostly generic}
\begin{itemize}
\item \texttt{if, else}: testing a condition [R, SAS] 
\item \texttt{for}: execute a loop a fixed number of times [R, in SAS: \texttt{do}]
\item \texttt{while}: execute a loop while a condition is true [R,SAS]
\item \texttt{until}: execute a loop until a condition is true [SAS]
\item \texttt{repeat}: execute an infinite loop [R]
\item \texttt{break}: break the execution of a loop [R, SAS]
\item \texttt{next}: skip an interation of a loop [R]
\item \texttt{return}: exit a function [R]
\end{itemize}
\end{block}
\end{frame}





\subsection{if}
\begin{frame}[fragile]{Control structures: if}
\begin{columns}
\begin{column}[t]{.48\textwidth}
\pause
\begin{block}{... in R}
\begin{verbatim}
if(<condition>) {
## do something
} else {
## do something else
}
if(<condition1>) {
## do something
} else if(<condition2>) {
## do something different
} else {
## do something different
}
\end{verbatim}
\end{block}
\end{column}
\hfill\pause
\begin{column}[t]{.48\textwidth}
\begin{block}{... in SAS}
\begin{verbatim}
if (<condition>) then do;
## do something
end; else do;
## do something else
end;
if (<condition1>) then do;
## do something
else if (<condition2>) then do;
## do something different
end; else do;
## do something different
end;
\end{verbatim}
\end{block}
\end{column}
\end{columns}
\end{frame}




\subsection{for}

\begin{frame}[fragile]{Control structures: for}
Run through a fixed sequence of numbers (or in R, a sequence of vectors)
\begin{columns}
\begin{column}[t]{.48\textwidth}
\pause
\begin{block}{simple loop in R}
\begin{verbatim}
for(i in 1:10) {
print(i)
}
\end{verbatim}
\end{block}
\end{column}
\hfill\pause
\begin{column}[t]{.48\textwidth}
\begin{block}{... in SAS}
\begin{verbatim}
do i = 1 to 10;
put i;
end;
\end{verbatim}
\end{block}
\end{column}

\end{columns}
\end{frame}


\begin{frame}[fragile]{Control structures: for}
Across programming languages, some flexibility:
\begin{columns}
\begin{column}[t]{.48\textwidth}
\pause
\begin{block}{Equivalent loops in R}
\begin{verbatim}
x <- c("a", "b", "c", "d")
for(i in 1:4) {
print(x[i])
}
for(i in x) {
print(i)
}
for(i in 1:4) print(x[i])
\end{verbatim}
\end{block}
\end{column}
\hfill
\begin{column}[t]{.48\textwidth}
\begin{block}{... in SAS}
\begin{verbatim}
do i = 1 to 10;
put i;
end;
\end{verbatim}
\end{block}
\end{column}

\end{columns}
\end{frame}








\end{document}