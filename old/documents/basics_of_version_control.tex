% Time-stamp: <09/10/02 01:57:13 vilhuber>
% $Id: Presentation-PSD.tex 3219 2012-09-27 07:47:11Z vilhu001 $

% normal line:
\documentclass[xcolor=table,compress]{beamer}
% to create notes:
%\documentclass[handout,notes=only]{beamer}
% to create handouts
%\documentclass[xcolor=table,handout,compress]{beamer}
% to create a different kind of handouts
%\documentclass{article}
%\usepackage[envcountsect]{beamerarticle}

%\setbeameroption{handout}
%\setbeameroption{show notes}


%
% Packages
%
\mode<article> % only for the article version
{
  \usepackage{fullpage}
  \usepackage{hyperref}
}
\usepackage{ifpdf}
\ifpdf
\usepackage{embedfile}
\embedfile{\jobname.tex}
\fi

\usepackage{graphicx}
%\usepackage{pstricks}
\usepackage{xcolor}
\usepackage{pifont}
%\usepackage{../chicago}
\usepackage{pgf}
\usepackage{amsmath,amssymb,amsfonts}
\usepackage[latin1]{inputenc}
\usepackage{colortbl}
\usepackage[english]{babel}
\usepackage{array}
\usepackage{pdfpages}
% usage:
%   \includepdf[pages={1}]{myfile.pdf}
%   \includepdf[pages={1,3,5}]{myfile.pdf} would include pages 1, 3, and 5 of the file. 
%   To include the entire file, you specify pages={-}, where {-}
%\usepackage{landscape}
\usepackage{listings}
\lstloadlanguages{R,bash}
\lstset{numbers=left, stepnumber=1,  language=bash, basicstyle=\tiny}

%\usepackage{lmodern}
%\usepackage[T1]{fontenc}

\usepackage{times}
%\usepackage{colortbl}

%============================================================
% Beamer specific styles and configs
%============================================================

\mode<presentation>
{
% alternative, could always use
%\usetheme{Census}
\usetheme{cornell}
\useoutertheme{cornell}
}


%\setbeamercovered{dynamic}



%============================================================
% Title
%============================================================

\title[Basics of VCS]{Basics of Version Control Systems}
\author[Vilhuber]{%
  Lars~Vilhuber\inst{1}
}

\institute[Cornell]{
  \inst{1}%
   Cornell University, Economics Department,
%\and \inst{2} U.S. Census Bureau
}%
\date[August 20-22, 2018]{August 20-22, 2018: Day 2}
\subject{HPC}


% % % % % % % % % % % % % % % % % Main document
\begin{document}
\frame{\titlepage}
\section{Intro}
\section{Basics}
\section[VCS]{Version control systems}


\subsubsection{Definition}

\begin{frame}{What are VCS}
\begin{block}{Software development perspective}
Version control systems (VCS) (or Source Configuration Management (SCM) systems $*$) allow developers or authors to keep track of the history of their projects? source code. [\href{http://better-scm.shlomifish.org/}{source}]

\end{block}
\pause
\begin{block}{Generic view}
Detailed mechanism to manage different versions (historical, parallel) of documents, files, programs, etc.
\end{block}
\end{frame}



\begin{frame}{You already have it}
\begin{block}{Implicit uses}
\begin{itemize}
\item Backup systems (Apple Time Machine, others)
\item Word processors (Undo feature, Track changes in Word, finer-grained features in content management systems: Google Docs, Blog software, etc.)
\item Versioning filesystems
\item Paper books!
\end{itemize}
\end{block}
\end{frame}



\begin{frame}{Modern Labor Economics: Theory and Public Policy}
\includegraphics[height=.7\textheight]{EhrenbergSmith.jpg}
\pause
\includegraphics[height=.7\textheight]{EhrenbergSmith-preface-11.png}
\end{frame}

\begin{frame}{Principal idea}
\begin{columns}

\begin{column}{.48\textwidth}
\includegraphics[height=.7\textheight]{Revision_controlled_project_visualization-2010-24-02.png}
\end{column}
\hfill
\begin{column}{.48\textwidth}
\begin{itemize}[<+->]
\item[1] First edition
\item[4] Second edition
\item[5] Start of work on a Canadian edition
\item[6] Start of work on the next US edition
\item[9] Third edition
\item[10] First Canadian edition
\end{itemize}
\end{column}
\end{columns}
\end{frame}



\begin{frame}[fragile]{File-system based versioning}
The most common method... Also used in email versioning...
\begin{block}{Remember this?}
\footnotesize \ttfamily
\begin{itemize}
\item[\ ] 01\_01\_readBLS.R
\item[\ ] {02\_01\_readCensus.R}
\item[\ ] 02\_02\_prepareCensus.R
\item[\ ] 03\_01\_create\_analysis\_data.R
\item[\ ] 04\_01\_runOLS.sas
\item[\ ] README.txt
\end{itemize}
\end{block}
\end{frame}



\begin{frame}[fragile]{File-system based versioning}
The most common method... Also used in email versioning...
\begin{block}{\onslide<1->{What if I make changes?}
}
\footnotesize \ttfamily
\begin{itemize}
\item[\ ] 01\_01\_readBLS.R
\item[\ ] \onslide<1->{02\_01\_readCensus.R}
\item[\ ] 02\_02\_prepareCensus.R
\item[\ ] 03\_01\_create\_analysis\_data.R
\item[\ ] 04\_01\_runOLS.sas
\item[\ ] README.txt
\onslide<2->{\item[\ ] 02\_01\_readCensus.R.bak}
\onslide<3->{\item[\ ] 02\_01\_readCensus\_V2.R}
\onslide<4->{\item[\ ] 02\_01\_readCensus\_V3.R}
\onslide<5->{\item[\ ] 02\_01\_readCensus\_V3-jma.R}
\onslide<6->{\item[\ ] 02\_01\_readCensus\_V3-jma-rm.R}
\end{itemize}
\end{block}
\end{frame}


\begin{frame}{Better way}
Is there a better way?
\end{frame}


\begin{frame}{History}
\includegraphics[width=.9\textwidth]{trac-svn-view2.png}
\end{frame}


\subsubsection[Types]{Types of VCS}
\begin{frame}{Two major types of version-management}
\begin{block}{Centralized model}
Server-client approach, editors check out a copy, modify it, and check it back in. Multiple editors:
\begin{itemize}
\item \textit{File locking}: only one person can check out any given file
\item \textit{Version merging}: discrepancies are handled upon checkin
\end{itemize}
\end{block}
\pause
\begin{block}{Distributed model}
There is no central server (prescribed by software). Every editor has a full copy of all version, synchronisation occurs by exchanging patches.
\end{block}
\end{frame}


\begin{frame}{Focus in this class}
We will focus on Git:
\begin{itemize}
\item Subversion (centralized) - still broadly in use, but no longer supported at Cornell 
\item Git (decentralized) - Hautahi
\begin{itemize}
	\item[Windows] \href{http://tortoisegit.org/}{TortoiseGit} (free)
	\item[OSX] installed (Xcode) upon first use
	\item[Linux] typically integrated
\end{itemize}
with various semi-specialized graphical versions (Github client, etc.)
\end{itemize}
\end{frame}




\begin{frame}{THE reference for Git}
\begin{center}
\includegraphics[height=0.7\textheight]{./progit2}\newline\Large
\href{https://git-scm.com/book/en/v2}{git-scm.com/book/en/v2}
\end{center}
\end{frame}


\begin{frame}{Learning more}
\begin{block}{Numerous resources on the web}
\begin{itemize}
	\item \url{https://try.github.io/}
	\item \url{https://www.atlassian.com/git/tutorials}
\end{itemize}
\end{block}
\end{frame}









\subsection{Tracking history}
\begin{frame}{Tracking history}
\begin{block}{What happened? And who did what?}
One of the key advantages of using version control systems is ... to control versions.
\begin{itemize}
\item Straightforward to view multiple versions of a file (assuming proper usage)
\item Possibility to view who changed what (``blame'' or ``annotate'')
\end{itemize}
\end{block}
\end{frame}

\begin{frame}{Tracking using web interfaces: SVN}
\includegraphics[width=.9\textwidth]{trac-svn-view1.png}
\end{frame}

\begin{frame}{Tracking using web interfaces: SVN}
\includegraphics[width=.9\textwidth]{trac-svn-view2.png}
\end{frame}

\begin{frame}{Tracking using web interfaces: SVN}
\includegraphics[width=.9\textwidth]{trac-svn-view3.png}
\end{frame}

\begin{frame}{Tracking using web interfaces: SVN}
\includegraphics[width=.9\textwidth]{trac-svn-view4.png}
\end{frame}


\begin{frame}{Tracking using web interfaces: git}
\includegraphics[width=.9\textwidth]{git-view1.png}
\end{frame}

\begin{frame}{Tracking using web interfaces: git}
\includegraphics[width=.9\textwidth]{git-view2.png}
\end{frame}

\begin{frame}{Tracking using web interfaces: git}
\includegraphics[width=.9\textwidth]{git-view3.png}
\end{frame}


%\subsection{Comparing systems}
%\begin{frame}[fragile]{Subversion vs. Git}
%\small
%incomplete
%\begin{block}{Checking out a subdirectory}
%\begin{columns}
%\begin{column}{.48\textwidth}
%\color{red}\rule{\linewidth}{4pt}
%
%SVN
%\end{column}%
%\hfill%
%\begin{column}{.48\textwidth}
%\color{blue}\rule{\linewidth}{4pt}
%
%GIT
%\end{column}%
%\end{columns}
%\begin{columns}
%\begin{column}{.48\textwidth}
%\color{red}
%\begin{lstlisting}
%svn co ${URL}/sub/directory (name)
%\end{lstlisting}
%\end{column}%
%\hfill%
%\begin{column}{.48\textwidth}
%\color{blue}\tiny
%\begin{lstlisting}
%mkdir <repo> && cd <repo>
%git init
%git remote add -f <name> <url>
%git config core.sparsecheckout  true
%echo some/dir/ >> .git/info/sparse-checkout
%echo another/sub/tree >>  .git/info/sparse-checkout
%git pull <remote> <branch>
%\end{lstlisting}
%{\tiny Source: \href{http://jasonkarns.com/blog/subdirectory-checkouts-with-git-sparse-checkout/}{here}}
%\end{column}%
%
%\end{columns}
%\end{block}
%\end{frame}


\begin{frame}{VCS infrastructure}
\begin{block}{At Cornell}
\href{https://forge.cornell.edu/sf/sfmain/do/home}{Cornell Sourceforge} (obsolete as of Sept 30, 2018)
\end{block}
\begin{block}{Elsewhere}
\begin{itemize}
\item \href{https://github.com/plans}{GitHub} (Git, subversion, free for open-source and academic users)
\item \href{http://bitbucket.org/}{BitBucket} (Git, no subversion, free for academic users)
\item Roll your own: no server required for Git on your PC
\end{itemize}
\end{block}
\end{frame}




\section{Cheat sheets}
\begin{frame}
\begin{itemize}
	\item \href{../external/git-cheatsheet-EN-dark.pdf}{Git command-line cheat sheet}
	\item \href{../external/xcode-cheat-sheet.pdf}{Xcode cheat sheet} (for Mac users)
		
\end{itemize}

\end{frame}

\end{document}