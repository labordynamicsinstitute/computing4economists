% Time-stamp: <09/10/02 01:57:13 vilhuber>
% $Id: Presentation-PSD.tex 3219 2012-09-27 07:47:11Z vilhu001 $

% normal line:
\documentclass[xcolor=table,compress]{beamer}
% to create notes:
%\documentclass[handout,notes=only]{beamer}
% to create handouts
%\documentclass[xcolor=table,handout,compress]{beamer}
% to create a different kind of handouts
%\documentclass{article}
%\usepackage[envcountsect]{beamerarticle}

%\setbeameroption{handout}
%\setbeameroption{show notes}


%
% Packages
%
\mode<article> % only for the article version
{
  \usepackage{fullpage}
  \usepackage{hyperref}
}
\usepackage{ifpdf}
\ifpdf
\usepackage{embedfile}
\embedfile{\jobname.tex}
\fi

\usepackage{graphicx}
%\usepackage{pstricks}
\usepackage{xcolor}
\usepackage{pifont}
%\usepackage{../chicago}
\usepackage{pgf}
\usepackage{amsmath,amssymb,amsfonts}
\usepackage[latin1]{inputenc}
\usepackage{colortbl}
\usepackage[english]{babel}
\usepackage{array}
\usepackage{pdfpages}
% usage:
%   \includepdf[pages={1}]{myfile.pdf}
%   \includepdf[pages={1,3,5}]{myfile.pdf} would include pages 1, 3, and 5 of the file. 
%   To include the entire file, you specify pages={-}, where {-}
%\usepackage{landscape}
\usepackage{listings}
\lstloadlanguages{R,bash}
\lstset{numbers=left, stepnumber=1,  language=bash, basicstyle=\tiny}

%\usepackage{lmodern}
%\usepackage[T1]{fontenc}

\usepackage{times}
%\usepackage{colortbl}

%============================================================
% Beamer specific styles and configs
%============================================================

\mode<presentation>
{
% alternative, could always use
%\usetheme{Census}
\usetheme{cornell}
\useoutertheme{cornell}
}


%\setbeamercovered{dynamic}



%============================================================
% Title
%============================================================

\title[Computing for Economists]{Workshop: High-performance computing for economists}
\author[Vilhuber, Abowd, Mansfield, McKinney]{%
  Lars~Vilhuber\inst{1} \and
  John M. Abowd\inst{1} \and
  Richard~Mansfield\inst{1} \and
  Kevin~L.~McKinney %\inst{2}%
}

\institute[Cornell]{
  \inst{1}%
   Cornell University, Economics Department,
%\and \inst{2} U.S. Census Bureau
}%
\date[August 20-22, 2013]{August 20-22, 2013: Day 3}
\subject{HPC}


% % % % % % % % % % % % % % % % % Main document
\begin{document}
\frame{\titlepage}

\section[HP resources]{HP resources at Cornell and elsewhere}
\begin{frame}
\href{day3-1.pdf}{HP resources}
\end{frame}
\section[Data mgmt]{Considerations for data management}
\begin{frame}
\href{day3-2.pdf}{Data management}
\end{frame}


\section[Basics]{Basics of High-performance computing}
\begin{frame}
\begin{block}{The gist of it all: QSUB}
\begin{itemize}
\item Basic submission
\item Understanding queue parameters
\item Customizing
\end{itemize}
\end{block}
\end{frame}


\begin{frame}[fragile]{What to watch out for}
\begin{itemize}
\item Too many jobs 
\item Within-job scheduling
\item Threading of your software (SAS, Stata, Matlab)
\item Storage
\begin{itemize}
\item Size
\item Speed
\end{itemize}
\end{itemize}
\end{frame}

\begin{frame}[fragile]{Wihin-job scheduling}
\begin{itemize}
\item Simple
\item \href{http://repository.vrdc.cornell.edu/VRDC/tools/trunk/pmon/}{Pmon} 
[locked]
\end{itemize}
\end{frame}

\begin{frame}[fragile]{Simple within-job scheduler}
\begin{lstlisting}[language=bash,numbers=none]
01_01_readBLS.R
02_01_readCensus.R
02_02_prepareCensus.R
03_01_create_analysis_data.R
04_01_runOLS.R
README.txt
\end{lstlisting}
\end{frame}

\begin{frame}[fragile]{Simple within-job scheduler}
\begin{lstlisting}[language=bash]
#!/bin/bash
#PBS -l ncpus=2
cd /to/mydir
set -- $(ls 0*R)
while [[ ! -z $1 ]] 
do
   R --vanilla < $1 &
   shift
   R --vanilla < $1 
   shift
   wait
done
\end{lstlisting}
\end{frame}


\begin{frame}{Threading}
\begin{block}{SAS}
\begin{itemize}
\item Default on ECCO set to 1 thread.
\item Override with "sas -cpucount 2"
\item ON ECCO, override with "qsas prog.sas [chunks]"
\end{itemize}\end{block}
\end{frame}


\begin{frame}[fragile]{qsas}
\begin{lstlisting}
> qsas
    /usr/local/bin/qsas prog[.sas] chunks

    will launch SAS under PBS-like systems, requesting [chunks] chunks
    and adjusting SAS memsize, sortsize, and sumsize appropriately.

    If not specifying [chunks], uses 1 CPU and 8GB of RAM.
    Chunks are defined in units of 2CPUS:16GB of RAM

    If these limits are insufficient, you may need to run a 
    custom qsub job with '#PBS -l mem=XXXXmb' as one of the PBS options.

    (expert usage)
    To add additional PBS options, set a environment variable PBSEXTRA
    with the full set of options. It will be appended to the qsub
    command line 
\end{lstlisting}
\end{frame}


\begin{frame}{Example job}
See QWI Macro examples
\begin{itemize}
\item 
\href{http://repository.vrdc.cornell.edu/private/jma7/qwi-macro.production/programs/}{code}
\item \href{http://www2.vrdc.cornell.edu/news/data/qwi-national-data/}{data}
\end{itemize}
\end{frame}

\begin{frame}[fragile]{Storage speed differences}
\begin{lstlisting}[language=SAS]
> iqsub 2
-sh-3.1$ echo "libname here '.'; 
               data here.one;
                 do i = 1 to 100000000;
                   output;
                 end;
               run; 
               proc datasets library=here; 
                 delete one;
               quit;
               " | sas -stdio 2>&1 | grep "real time" | tail -1

      real time           6.30 seconds

-sh-3.1$ cd /dev/shm
-sh-3.1$ echo "libname here '.'; data here.one;do i = 1 to 
100000000;output;end;run; proc datasets library=here; delete one;quit;" | sas 
-stdio 2>&1 | grep "real time" | tail -1

      real time           3.38 seconds

-sh-3.1$ cd /temporary/
-sh-3.1$ echo "libname here '.'; data here.one;do i = 1 to 
100000000;output;end;run; proc datasets library=here; delete one;quit;" | sas 
-stdio 2>&1 | grep "real time" | tail -1

      real time           4.27 seconds
\end{lstlisting}
\end{frame}

\section{Wrap up}

\begin{frame}
\href{day3-4.pdf}{Wrap-up}
\end{frame}

\end{document}